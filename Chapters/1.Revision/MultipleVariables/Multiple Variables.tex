\documentclass{article}
\graphicspath{{/home/david/Book/Chapters/1.Revision/LinearEquation/pic/}}

% !TeX root = ../../../Mainfile/book.tex



\begin{document}

\subsection{Linear equation with multiple variables}

\begin{equation} \label{eq:2}
x + 9y + z = 3
\end{equation}

Lets now extend our definition of a linear equation to include more variables. \eqref{eq:1} had only $x$ and $y$, but \eqref{eq:2} has 3 variables, $x$, $y$ and $z$. 

The $9$ in front of the $y$? That is called a \textit{coefficient}, and the $3$ on the right-hand-side is called a \textit{constant}

\

Because we will (eventually) run out of letters in the alphabet, we write our equations like this:

\[
\color{rooj}c_1\color{blou}x_1\color{black} + \color{rooj}c_2\color{blou}x_2\color{black} + \color{rooj}c_3\color{blou}x_3\color{black} + \dots\color{black} + \color{rooj}c_n\color{blou}x_n\color{black} = \color{groen} b \color{black}
\]

Here the $c$'s are the \color{rooj} \textit{coefficients}\color{black}, the $x$'s are the \color{blou}  \textit{variables} \color{black}  and the $b$ is (spoiler) the \color{groen} \textit{constant} \color{black} . 

It may look complex the first time, but you'll get used to reading equations like this.

\paragraph{Lets look at an example:}

Jimmy goes to the store to buy cokes, snickers and apples. Jimmy knows that a can of coke is $2.2\$$, snickers is $1.6\$$ and an apple is $3.0\$$

If $c_1$ is price of coke, $c_2$ is price of snickers and $c_3$ is price of apples, our equation would look like this

\[
2.2x_1 + 1.6x_2 + 3.0x_3 = b
\]

Jimmy needs a couple of cokes ($x_1$) and four apples ($x_3$). Jimmy has $20\$$.

\textit{How many snickers bars can he buy with the leftover money?}

\[
2.2\cdot 2 + 1.6x_2 + 3.0\cdot 4 = 20
\]

Do the math and help Jimmy get his snickers by solving for $x_2$. 


\end{document}