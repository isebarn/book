\documentclass{article}
\graphicspath{{/home/david/Book/Chapters/1.Revision/LinearEquation/pic/}}

% !TeX root = ../../../Mainfile/book.tex



\begin{document}


\subsection{System of equations}

\begin{align*}
& 2x_1 + x_2 = 17\\
& x_1 + 3x_2 = 26
\end{align*}

The equations above are an example of a \textbf{system of equations}.

We want to solve these systems by finding the correct values for the \textit{variables}, in this case, $x_1$ and $x_2$, so that both equations work out.

We could start by \textit{guessing} that $x_1 = 3$ and $x_2 = 5$, which would give us 

\begin{align*}
& 2\cdot 3 + 2 = 8 \color{rooj} \neq 17\\
& \color{black} 3 + 3\cdot 5 = 18 \color{rooj} \neq 26 \color{black}
\end{align*}

This is far from correct, we need the \textbf{the substitution method}.

The substitution method consists of \textbf{two} steps, that you use over and over again until the system has been solved. These steps are:

\begin{enumerate}
\color{blou} \item Isolating a variable 
\color{rooj} \item Substitution 
\color{groen} \item Simplification
\end{enumerate}

Lets take another look at our system

\begin{align}
& 2x_1 + x_2 = 17\\ 
& x_1 + 3x_2 = 26 
\end{align}

and solve it using the substitution method.

\begin{enumerate}
\item \color{blou} isolate the $x_1$ from (4), $x_1 = 26 - 3x_2$  \color{black}
\item  \color{rooj} substitute $x_1$ into (3), $2 (26 - 3x_2) + x_2 = 17$ \color{black}
\item now the system looks like
\begin{align}
& 2 (26 - 3x_2) + x_2 = 17 \\ 
& x_1 + 3x_2 = 26      
\end{align}
\item \color{groen} which simplifies to  \color{black}
\begin{align}
& x_2 = 7 \\ 
& x_1 + 3x_2 = 26      
\end{align}
\item \color{rooj} now we just insert $x_2$ into (8)  to get \color{black}
\begin{align}
& x_2 = 7 \\ 
& x_1 + 21 = 26      
\end{align}
\item \color{groen} so (8) simplifies to $\mathbf{x_1 = 5}$ \color{black}

\end{enumerate}

Now its your turn to solve the following 

\begin{align*}
& 6x_1 + 2x_2 = 70 \\
& 3x_1 + 3x_2 = 45 
\end{align*}


\end{document}