\documentclass{article}


\graphicspath{{/home/david/Book/Chapters/2.Vectors/DotProduct/pic/}}
% !TeX root = ../../../Mainfile/book.tex



\begin{document}

\color{white}
\subsection{Vector dot product}
\color{black}

\begin{equation*}
\vect{1,3} \bullet \vect{4,1} = 1\cdot 4 +3\cdot 1 = 7
\end{equation*}



\paragraph{Dot product:} When taking a dot product of two vectors we multiply their corresponding elements together and then add up the products. The result is a single number. 

\color{theorem} \paragraph{Definition:} \textit{Say we have two vectors $u=\vect{u_1, u_2,\dots,u_n}$ and $v=\vect{v_1,v_2,\dots,v_n}$, their dot product is defined as 
\[
\vec{u}\bullet \vec{v} = u_1\cdot v_1 + u_2\cdot v_2+ \dots u_n\cdot v_n 
\]
} \color{black}

\paragraph{Uses:} The dot product tells us a lot about the direction of the vectors relative to each other:
\begin{itemize}
\item A positive dot product: The angle between the vectors is less than $90^\circ$. 
\item The dot product is 0: The vectors are perpendicular to each other. 
\item A negative dot product: The angle between the vectors is more than $90^\circ$. 
\end{itemize}

The dot product can even help us find the exact angle between the vectors. 

\paragraph{Example:} "We can use \textbf{this} to calculate \textbf{that}", followed by an example

\paragraph{Mynd?} Always try to have a visual along with the example

\paragraph{Exercise:} Exercise for the reader


		%2.3	Vector dot product 
%
			%What is it/how does it work? 
			%Result is number
			%What does it tell us? 0 = perpendicular


\end{document}