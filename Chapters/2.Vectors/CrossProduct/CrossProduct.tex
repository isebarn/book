\documentclass{article}

\graphicspath{{../Chapters/2.Vectors/CrossProduct/pic/}}
% !TeX root = ../../../Mainfile/book.tex



\begin{document}

\color{white}
\subsection{Cross Product}
\color{black}

\begin{equation*}
\vect{1,3,2}\times \vect{4,1,1} 
\end{equation*}

\paragraph{Mynd} 

\paragraph{Cross Product:} When taking the cross product of two vectors, $\vec{u}$ and $\vec{v}$, the result is a \textbf{vector} that is perpendicular to both $\vec{u}$ and $\vec{v}$. Also, the length of the cross product vector is equal to the area of the parallelogram bordered by $\vec{u}$ and $\vec{v}$ as shown on the diagram above.  

\color{theorem} \paragraph{Definition:} \textit{The Cross Product of two 3-dimensional vectors, $\vec{u} = \vect{u_1, u_2, u_3}$ and $\vec{v}=\vect{v_1,v_2,v_3}$}, is defined as the vector
\begin{equation}\label{eq:13}
\vec{u}\times\vec{v}= \vect{u_2\cdot v_3-u_3\cdot v_2\quad, u_3\cdot v_1-u_1\cdot v_3\quad, u_1\cdot v_2-u_2\cdot v_1}
\end{equation}
\color{black}  

\paragraph{Uses:} As mentioned above the cross product is a vector perpendicular to both vectors which is useful in many scenarios, for example ....

Area. 

\paragraph{Example:} Let's calculate the cross product of $\vec{u}=\vect{1,7,5}$ and $\vec{v}=\vect{2,11,4}$. We simply plug the correct values into \ref{eq:13}

\begin{align*}
\vect{1,7,5}\times\vect{2,11,4} &= \vect{7\cdot 4 - 5\cdot11\quad, 5\cdot2 - 1\cdot4\quad, 1\cdot11 - 7\cdot2}\\
&=\vect{-27, 6, -3 }
\end{align*}

This means that the area of the parallelogram bordered by $\vec{u}$ and $\vec{v}$ is 

\begin{align*}
|\vec{u}\times \vec{v}| &= |\vect{-27, 6,-3}|\\
&= \sqrt{(-27)^2+6^2+(-3)^2}\\
&=\sqrt{774}\\
&\approx 27.82
\end{align*}


\paragraph{Exercise:} Do the same for $\vec{u} = \vect{6,3,1}$ and $\vec{v} = \vect{0,2,5}$.


%Only 3d vectors



\end{document}